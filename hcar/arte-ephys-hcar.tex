\documentclass[DIV16,twocolumn,10pt]{scrreprt}
\usepackage{paralist}
\usepackage{graphicx}
\usepackage[final]{hcar}

%include polycode.fmt

\begin{document}

\begin{hcarentry}{(arte-ephys)}
\report{(Greg Hale)}
\status{(Work in progress)}
\participants{(Alex Chen, Sarah Walker)}% optional
\makeheader

arte-ephys is a soft realtime neural recording system for tetrodes.  

Several labs implant electrodes into animals to record the signals emitted by neurons.  These signals are often recorded from ten to thirty electrodes in parallel in people and freely moving animals; and the raw data go through manual analysis to decode the activity of roughly one hundred neurons whose signals reach the electrodes, and to determine how these neurons encode stimuli in the environment.



Put the text here. 
If you want to include Haskell code, consider using lhs2tex syntax (\url{http://people.cs.uu.nl/andres/lhs2tex/}).

What's following are suggestions for the content of an entry.

(WHAT IS IT?)

(WHAT IS ITS STATUS? / WHAT HAS HAPPENED SINCE LAST TIME?)

(CAN OTHERS GET IT?)

(WHAT ARE THE IMMEDIATE PLANS?)

\FurtherReading
  \url{(PROJECT URL)}
\end{hcarentry}

\end{document}
